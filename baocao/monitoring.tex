% -----------------------------------------------------------------
% 7.1 Monitoring
%
% Văn phong: Kỹ sư MLOps - tập trung vào observability, metrics, alerting
% Tổ chức: System Metrics → Data Metrics → Alerting → Decision Framework

\textit{Monitoring là thành phần không thể thiếu trong vòng đời vận hành hệ thống gợi ý.
Bên cạnh việc đảm bảo \textbf{service availability} và \textbf{performance SLA},
monitoring còn đóng vai trò then chốt trong việc phát hiện \textbf{data drift} ---
hiện tượng phân phối dữ liệu thay đổi theo thời gian khiến mô hình mất hiệu quả.
Phần này trình bày chi tiết các chỉ số giám sát và cơ chế cảnh báo, được implement trong
\texttt{alerting.py}, \texttt{automation/drift\_detection.py}, và \texttt{service/dashboard.py}.}

% ===================================================================
\subsubsection*{System Metrics: Giám sát hiệu năng dịch vụ}
% ===================================================================

System metrics đo lường trực tiếp hành vi của service ở mức infrastructure,
cho phép phát hiện sớm các vấn đề về tải, lỗi, và tắc nghẽn.

\paragraph{Latency Quantiles.}

Độ trễ phản hồi được theo dõi qua các quantile:
\begin{itemize}
  \item \textbf{P50} (median): Latency ``điển hình'' mà người dùng trải nghiệm.
  \item \textbf{P90}: 90\% requests hoàn thành nhanh hơn giá trị này.
  \item \textbf{P95}: Chỉ số SLA chính --- target $< 100$ms.
  \item \textbf{P99}: Phát hiện ``tail latency'' bất thường.
\end{itemize}

Công thức tính percentile $P_k$ từ tập quan sát $\{x_1, \dots, x_n\}$ (đã sắp xếp):
\[
  P_k = x_{\lceil k \cdot n / 100 \rceil}
\]

Ngưỡng cảnh báo được cấu hình trong \texttt{config/alerts\_config.yaml}:

\begin{center}
\begin{tabular}{|l|c|c|c|}
\hline
\textbf{Metric} & \textbf{Warning} & \textbf{Critical} & \textbf{Window} \\
\hline
\texttt{avg\_latency\_ms} & $> 200$ms & $> 400$ms & 5 min \\
\texttt{p95\_latency\_ms} & --- & $> 500$ms & 5 min \\
\hline
\end{tabular}
\end{center}

\paragraph{Error Rate.}

Tỷ lệ lỗi được định nghĩa là phần trăm requests trả về HTTP 4xx/5xx:
\[
  \text{Error Rate} = \frac{|\{r : r.\text{status} \geq 400\}|}{|\text{total requests}|}
\]

Ngưỡng:
\begin{itemize}
  \item \textbf{Warning}: Error rate $> 1\%$ trong cửa sổ 5 phút.
  \item \textbf{Critical}: Error rate $> 5\%$ --- trigger incident response.
\end{itemize}

\paragraph{Fallback Rate.}

Tỷ lệ requests được xử lý bởi fallback path (content-based thay vì CF):
\[
  \text{Fallback Rate} = \frac{|\{r : r.\text{fallback} = \text{true}\}|}{|\text{total requests}|}
\]

Với thiết kế hệ thống (91.4\% cold-start users), fallback rate kỳ vọng $\approx 91\%$.
Nếu fallback rate $> 95\%$ --- có thể CF model bị lỗi hoặc mapping không đồng bộ.

\paragraph{Throughput.}

Số requests được xử lý mỗi phút (\texttt{requests\_per\_minute}):
\begin{itemize}
  \item \textbf{Target}: $\geq 100$ req/s (6,000 req/min).
  \item \textbf{Alert}: $< 50$ req/s trong giờ cao điểm.
  \item \textbf{Low traffic}: 0 requests trong 5 phút --- kiểm tra upstream.
\end{itemize}

% ===================================================================
\subsubsection*{Data Metrics: Phát hiện trôi dạt dữ liệu (Data Drift)}
% ===================================================================

Data drift là hiện tượng phân phối dữ liệu thay đổi theo thời gian, khiến mô hình
được huấn luyện trên dữ liệu cũ trở nên kém hiệu quả trên dữ liệu mới.
Hệ thống giám sát 4 loại drift chính.

\paragraph{Rating Distribution Drift.}

Hệ thống so sánh phân phối rating giữa dữ liệu lịch sử (baseline) và dữ liệu hiện tại.
Từ \texttt{drift\_detection.py}, sử dụng tổng hiệu phân phối:
\[
  \text{total\_diff} = \sum_{r=1}^{5} |P_{\text{current}}(r) - P_{\text{baseline}}(r)|
\]

\textbf{Quy tắc quyết định} (từ config):
\begin{itemize}
  \item Ngưỡng: \texttt{rating\_dist\_threshold = 0.1}
  \item Nếu \texttt{total\_diff} $> 0.1$: Drift được phát hiện $\rightarrow$ xem xét retrain.
\end{itemize}

\begin{verbatim}
result = detect_rating_drift(current_stats, baseline_stats, logger)
# Output: {'metric': 'rating_distribution', 
#          'drift_detected': True, 'total_difference': 0.15,
#          'details': {'rating_5_diff': 0.08, ...}}
\end{verbatim}

\paragraph{Popularity Shift.}

Sử dụng \textbf{Jaccard similarity} để so sánh top-20 items phổ biến nhất:
\[
  \text{Jaccard} = \frac{|\text{Top20}_{\text{current}} \cap \text{Top20}_{\text{baseline}}|}
                        {|\text{Top20}_{\text{current}} \cup \text{Top20}_{\text{baseline}}|}, \quad
  \text{shift} = 1 - \text{Jaccard}
\]

\textbf{Ngưỡng} (từ \texttt{DRIFT\_CONFIG}):
\begin{itemize}
  \item \texttt{popularity\_shift\_threshold = 0.2}
  \item Nếu \texttt{shift} $> 0.2$ --- popularity distribution đã thay đổi đáng kể,
        có thể do xu hướng mới, mùa vụ, hoặc viral products.
\end{itemize}

\begin{verbatim}
result = detect_popularity_drift(current, baseline, logger)
# Output: {'metric': 'popularity_distribution',
#          'drift_detected': False, 'jaccard_similarity': 0.85,
#          'shift': 0.15, 'details': {'new_items': [...]}}
\end{verbatim}

\paragraph{Interaction Rate Drift.}

Theo dõi thay đổi trong hành vi người dùng thông qua tỷ lệ tương tác trung bình:
\begin{itemize}
  \item \textbf{avg\_interactions\_per\_user}: Từ \texttt{data\_stats.json}
  \item \textbf{Change rate}: $\Delta = |\mu_{\text{current}} - \mu_{\text{baseline}}| / \mu_{\text{baseline}}$
\end{itemize}

\textbf{Ngưỡng}:
\[
  \texttt{interaction\_rate\_threshold} = 0.3 \quad (30\%)
\]

Nếu $\Delta > 0.3$: Interaction behavior đã drift đáng kể.

\begin{verbatim}
result = detect_interaction_drift(current, baseline, logger)
# Output: {'metric': 'interaction_rate', 'drift_detected': False,
#          'current_rate': 14.2, 'baseline_rate': 13.8,
#          'change_rate': 0.029}
\end{verbatim}

\paragraph{Embedding Freshness.}

\textbf{Vietnamese Embedding} (\texttt{AITeamVN/Vietnamese\_Embedding}, 1024 dim) cần được cập nhật định kỳ để phản ánh:
\begin{itemize}
  \item Sản phẩm mới được thêm vào catalog.
  \item Mô tả sản phẩm được cập nhật (ingredients, features).
  \item Cải tiến trong pre-trained model (nếu có).
\end{itemize}

\textbf{Chỉ số freshness} (từ \texttt{alerts\_config.yaml}):
\[
  \text{Age}_{\text{days}} = \lfloor \text{now} - \text{embeddings.created\_at} \rfloor
\]

\textbf{Ngưỡng}: \texttt{embedding\_age\_days = 30} $\rightarrow$ Embeddings được coi là \textit{stale}.

\textbf{Embedding Drift Detection}:
Khi regenerate embeddings, so sánh với version cũ bằng cosine similarity:
\[
  \text{sim}(i) = \frac{\mathbf{e}_i^{\text{old}} \cdot \mathbf{e}_i^{\text{new}}}
                      {\|\mathbf{e}_i^{\text{old}}\| \|\mathbf{e}_i^{\text{new}}\|}
\]

Nếu $\text{mean}(\text{sim}) < 0.95$: Embeddings đã drift đáng kể --- cần retrain CF model (BERT-ALS).

% ===================================================================
\subsubsection*{Comprehensive Drift Report (từ \texttt{generate\_drift\_report()})}
% ===================================================================

Hệ thống tổng hợp tất cả drift checks vào một report định kỳ (weekly), lưu tại 
\texttt{reports/drift/drift\_report\_YYYYMMDD\_HHMMSS.json}:

\begin{verbatim}
{
  "generated_at": "2025-11-27T10:00:00",
  "git_commit": "abc123",
  "drift_detected": true,
  "results": [
    {
      "metric": "rating_distribution",
      "drift_detected": true,
      "total_difference": 0.15,
      "threshold": 0.1
    },
    {
      "metric": "popularity_distribution", 
      "drift_detected": false,
      "jaccard_similarity": 0.85,
      "shift": 0.15,
      "threshold": 0.2
    },
    {
      "metric": "interaction_rate",
      "drift_detected": false,
      "change_rate": 0.029,
      "threshold": 0.3
    }
  ]
}
\end{verbatim}

Quy tắc tổng hợp:
\[
  \text{drift\_detected} = \bigvee_{\text{result } r} r.\text{drift\_detected}
\]

\paragraph{CLI Usage:}
\begin{verbatim}
# Run drift detection
python -m automation.drift_detection

# Update baseline after confirmed changes
python -m automation.drift_detection --update-baseline
\end{verbatim}

% ===================================================================
\subsubsection*{Alerting System}
% ===================================================================

Hệ thống alerting hỗ trợ 3 channels: \textbf{Log}, \textbf{Email}, và \textbf{Slack},
với severity levels: \texttt{info}, \texttt{warning}, \texttt{critical}.

\paragraph{Alert Definitions (từ \texttt{alerts\_config.yaml}).}

\begin{center}
\begin{tabular}{|l|l|l|l|}
\hline
\textbf{Alert} & \textbf{Condition} & \textbf{Severity} & \textbf{Action} \\
\hline
\texttt{high\_latency} & $\text{avg\_latency} > 200$ms & warning & log \\
\texttt{critical\_latency} & $\text{p95\_latency} > 500$ms & critical & email+slack \\
\texttt{high\_error\_rate} & $\text{error\_rate} > 5\%$ & critical & email+slack \\
\texttt{high\_fallback\_rate} & $\text{fallback} > 95\%$ & warning & log \\
\texttt{low\_traffic} & $\text{requests\_per\_minute} = 0$ & warning & log \\
\texttt{data\_drift} & drift\_detected = true & info & email \\
\hline
\end{tabular}
\end{center}

\paragraph{AlertManager Class (từ \texttt{alerting.py}).}

\begin{verbatim}
class AlertManager:
    def send_alert(self, subject, message, severity, metadata=None):
        # Log to logs/service/alerts.log
        # Email for warning/critical (if enabled)
        # Slack for warning/critical (if enabled)
        
    def check_alert_conditions(self, metrics, model_id):
        # Check all thresholds from config
        # Return list of triggered alerts
\end{verbatim}

\paragraph{Alert Cooldown.}

Để tránh alert fatigue, áp dụng cooldown period:
\[
  \text{send\_alert} = (\text{now} - \text{last\_alert\_time}) > T_{\text{cooldown}}
\]
với $T_{\text{cooldown}} = 30$ phút cho cùng một loại alert.

\paragraph{Alert Template (từ \texttt{templates} section).}

\begin{verbatim}
🚨 Critical Error Rate Alert

Current error rate: {value:.1%}
Threshold: {threshold:.1%}
Errors in last 5 minutes: {error_count}
Model: {model_id}

Action Required: Check service logs for errors.
\end{verbatim}

\paragraph{Convenience Functions (từ \texttt{alerting.py}):}
\begin{verbatim}
# Send alert using singleton AlertManager
from alerting import (send_alert, alert_high_latency, 
                       alert_high_error_rate, alert_data_drift,
                       alert_model_performance)

send_alert("High Latency", "P95 = 250ms", severity="warning")
alert_high_latency(latency_ms=250, threshold=200)
alert_high_error_rate(error_rate=0.08, threshold=0.05)
alert_data_drift(drift_result={'drift_detected': True, 'p_value': 0.02})
alert_model_performance(current_recall=0.18, baseline_recall=0.22, 
                        threshold_drop=0.1)
\end{verbatim}

% ===================================================================
\subsubsection*{Retrain Decision Framework}
% ===================================================================

Quyết định retrain được đưa ra dựa trên tổng hợp nhiều tín hiệu (từ \texttt{retrain\_triggers} trong config):

\begin{enumerate}
  \item \textbf{Rating Drift}: Total distribution difference $> 0.1$.
  \item \textbf{Popularity Shift}: Jaccard similarity $< 0.8$ (shift $> 0.2$).
  \item \textbf{Interaction Rate Change}: Change rate $> 30\%$.
  \item \textbf{Data Staleness}: Training data $> 30$ ngày tuổi.
  \item \textbf{Embedding Staleness}: Vietnamese Embeddings $> 30$ ngày tuổi.
  \item \textbf{Performance Drop}: CTR giảm $> 10\%$ so với baseline.
\end{enumerate}

\textbf{Decision Rule}:
\[
  \text{should\_retrain} = \bigvee_{i=1}^{6} \text{trigger}_i
\]

Nếu bất kỳ trigger nào active $\rightarrow$ đưa ra khuyến nghị retrain kèm lý do cụ thể.

\begin{verbatim}
# From alerts_config.yaml
retrain_triggers:
  rating_drift_pvalue: 0.05
  popularity_correlation: 0.8
  data_age_days: 30
  ctr_drop_percent: 10
  embedding_age_days: 30
\end{verbatim}

% ===================================================================
\subsubsection*{Dashboard và Visualization (\texttt{service/dashboard.py})}
% ===================================================================

Hệ thống cung cấp Streamlit dashboard với 5 tabs chính:

\begin{enumerate}
  \item \textbf{Service Health}: Real-time latency, error rate, fallback rate, requests/minute charts.
  \item \textbf{Training History}: Lịch sử training runs, model performance comparison (bar charts).
  \item \textbf{Scheduler}: Quản lý scheduled jobs (enable/disable, run now, view logs).
  \item \textbf{Drift Detection}: Kết quả drift checks, trend visualization.
  \item \textbf{Model Info}: Thông tin model đang serve, registry status.
\end{enumerate}

\paragraph{Data Sources:}
\begin{itemize}
  \item \textbf{Training DB}: \texttt{logs/training\_metrics.db} (SQLite) --- lưu training runs (table \texttt{training\_runs}) và iteration metrics (table \texttt{iteration\_metrics} với các cột \texttt{run\_id}, \texttt{iteration}, \texttt{loss}, \texttt{validation\_ndcg})
  \item \textbf{Service DB}: \texttt{logs/service\_metrics.db} (SQLite) --- lưu request logs (table \texttt{requests}) và health metrics (table \texttt{service\_health})
  \item \textbf{API}: \texttt{/scheduler/status}, \texttt{/scheduler/jobs} --- real-time scheduler status
\end{itemize}

Dashboard hỗ trợ:
\begin{itemize}
  \item \textbf{Time range selection}: Last 15min / 1h / 6h / 24h.
  \item \textbf{Auto-refresh}: Tự động cập nhật mỗi 10 giây (configurable).
  \item \textbf{Interactive charts}: Line charts cho metrics time series, bar charts cho model comparison.
\end{itemize}

\paragraph{Kết nối API:}
\begin{verbatim}
API_BASE_URL = os.environ.get("API_URL", "http://viecomrec-api:8000")
# Docker: viecomrec-api:8000
# Local: localhost:8000
\end{verbatim}

% ===================================================================
\subsubsection*{Check Intervals và Scheduling (từ \texttt{alerts\_config.yaml})}
% ===================================================================

\begin{center}
\begin{tabular}{|l|l|l|}
\hline
\textbf{Check Type} & \textbf{Interval} & \textbf{Rationale} \\
\hline
Health check & 60 giây & Phát hiện nhanh service issues \\
Drift detection & 168 giờ (weekly) & Tránh false positives từ noise \\
Embedding freshness & 24 giờ (daily) & Kiểm tra định kỳ \\
Alert cooldown & 30 phút & Giảm alert fatigue \\
\hline
\end{tabular}
\end{center}

\paragraph{Configuration (từ \texttt{intervals} section):}
\begin{verbatim}
intervals:
  health_check_seconds: 60
  drift_check_hours: 168
  alert_cooldown_minutes: 30
\end{verbatim}

Lịch trình monitoring được tích hợp với Automation Pipeline (mục 7.2)
để tự động trigger retrain khi drift được phát hiện.

\paragraph{Lưu ý về Embedding Models:}
Kiểm tra freshness cho Vietnamese Embedding (1024 dim) với ngưỡng 30 ngày.
Chi tiết về sự phân biệt với ViSoBERT được trình bày tại \textbf{mục 3.0.3.4}.
