% -----------------------------------------------------------------
% Kết luận và Hướng phát triển
%
% Văn phong: Tổng kết ngắn gọn, súc tích, highlight đóng góp chính

\textit{Báo cáo đã trình bày một hệ thống gợi ý sản phẩm mỹ phẩm hoàn chỉnh,
từ xử lý dữ liệu đến triển khai production. Phần này tổng kết các đóng góp
chính và định hướng nghiên cứu tiếp theo.}

% ===================================================================
\subsection{Kết luận}
% ===================================================================

Hệ thống đã giải quyết thành công các thách thức đặc thù của bài toán:

\paragraph{1. Xử lý dữ liệu thưa (Sparsity).}
Với chỉ 8.6\% người dùng có đủ dữ liệu huấn luyện, hệ thống áp dụng chiến lược
\textbf{dual-path routing}: CF cho trainable users và content-based fallback
(PhoBERT similarity + popularity) cho cold-start users --- đảm bảo 100\% requests
nhận được recommendations.

\paragraph{2. Khắc phục rating skew.}
Thay vì dùng rating thô (95\% là 5 sao), hệ thống tính \textbf{sentiment-enhanced
confidence score} kết hợp rating với chất lượng bình luận, giúp mô hình phân biệt
được các tương tác ``thật sự tích cực'' với các rating mặc định.

\paragraph{3. Tận dụng ngữ nghĩa tiếng Việt.}
\textbf{BERT Initialization} sử dụng PhoBERT embeddings để khởi tạo item factors,
giải quyết cold-start cho sản phẩm mới và cải thiện chất lượng embedding trong
không gian latent thưa.

\paragraph{4. Kiến trúc production-ready.}
Hệ thống đạt các SLA targets: latency P95 $< 100$ms, availability $\geq 99.9\%$,
với automation pipeline đảm bảo data freshness và model updates tự động.

% ===================================================================
\subsection{Hướng phát triển}
% ===================================================================

\paragraph{Ngắn hạn.}
\begin{itemize}
  \item \textbf{A/B Testing Framework}: So sánh hiệu quả thực tế giữa các chiến lược
  reranking (CTR, conversion rate).
  \item \textbf{Real-time Features}: Tích hợp session-based signals (recently viewed,
  cart items) vào scoring.
\end{itemize}

\paragraph{Trung hạn.}
\begin{itemize}
  \item \textbf{Graph Neural Networks}: Mô hình hóa quan hệ user-item-attribute
  dưới dạng heterogeneous graph, tận dụng thông tin cấu trúc (cùng brand, cùng skin type).
  \item \textbf{Multi-task Learning}: Đồng thời tối ưu click prediction và purchase
  prediction để cân bằng exploration-exploitation.
\end{itemize}

\paragraph{Dài hạn.}
\begin{itemize}
  \item \textbf{Reinforcement Learning}: Áp dụng contextual bandits hoặc Q-learning
  để học policy gợi ý tối ưu từ user feedback theo thời gian thực.
  \item \textbf{Explainable Recommendations}: Sinh giải thích tự nhiên cho từng gợi ý
  (``Vì bạn thích sản phẩm X có thành phần Y...'') sử dụng LLM.
\end{itemize}
