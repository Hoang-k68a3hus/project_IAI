\chapter{Thiết kế hệ thống và xử lý dữ liệu}

\section{Kiến trúc tổng quan hệ thống}

Hệ thống thương mại điện tử RabbitMart kết hợp dịch vụ gợi ý mỹ phẩm VieComRec được thiết kế theo mô hình \textbf{Client--Server} nhiều tầng, tích hợp thêm một \textbf{Recommender Service} chạy độc lập bằng Docker. Kiến trúc tổng thể tuân theo mô hình triển khai thực tế của các ứng dụng thương mại điện tử hiện đại:

\begin{itemize}
    \item \textbf{Frontend (Client Layer)}: Giao diện ReactJS 18 tương tác trực tiếp với người dùng, chạy trên cổng 3000.
    \item \textbf{Backend (Service Layer)}: Máy chủ Node.js/Express xử lý nghiệp vụ, xác thực người dùng và kết nối database, chạy trên cổng 5000.
    \item \textbf{Database Layer}: MongoDB 6.0 lưu trữ toàn bộ dữ liệu người dùng, sản phẩm, đơn hàng và lịch sử hành vi, chạy trên cổng 27017.
    \item \textbf{Recommendation Layer (VieComRec)}: API gợi ý sản phẩm chuyên biệt sử dụng FastAPI, được triển khai dưới dạng Docker service độc lập, chạy trên cổng 8000.
\end{itemize}

Các thành phần giao tiếp qua các endpoint RESTful, trong đó Backend đóng vai trò \textit{API Gateway}, điều phối dữ liệu giữa Client, MongoDB và VieComRec API. Hệ thống hỗ trợ cả người dùng đã đăng nhập và khách vãng lai (guest users).

\begin{figure}[H]
    \centering
    \includegraphics[width=0.9\textwidth]{image/architecture_diagram.png}
    \caption{Sơ đồ kiến trúc tổng quan hệ thống RabbitMart + VieComRec}
\end{figure}

\vspace{1em}

\section{Các thành phần chi tiết}

\subsection{Phân hệ Frontend (ReactJS)}

Phân hệ Client được xây dựng theo kiến trúc SPA (Single Page Application) sử dụng React 18 nhằm tối ưu tốc độ tải trang và trải nghiệm người dùng.

\subsubsection*{Công nghệ sử dụng}

\begin{itemize}
    \item \textbf{React 18.1.0}: Thư viện UI chính với hooks và functional components.
    \item \textbf{Redux Toolkit 1.8.1}: Quản lý state toàn cục (authentication, products).
    \item \textbf{React Router DOM 6.3.0}: Điều hướng SPA với các routes động.
    \item \textbf{Axios 0.27.2}: HTTP client giao tiếp với Backend APIs.
    \item \textbf{Framer Motion 6.3.3}: Animation và transitions cho UI.
\end{itemize}

\subsubsection*{Cấu trúc thư mục Frontend}

\begin{verbatim}
client/src/
├── api/                    # API clients
│   ├── index.js           # Backend API calls
│   ├── viecomrec.js       # VieComRec API client
│   └── BaseURLs.js        # Base URL configurations
├── actions/               # Redux actions
├── reducers/              # Redux reducers
├── components/            # Reusable components
│   ├── navigation/        # Navigation bar
│   ├── product-card/      # Product display card
│   ├── loading/           # Loading spinner
│   └── pages/             # Pagination component
├── pages/                 # Route pages
│   ├── home/              # Trang chủ
│   ├── products/          # Danh sách sản phẩm
│   ├── product-detail/    # Chi tiết sản phẩm
│   ├── cart/              # Giỏ hàng
│   ├── checkout/          # Thanh toán
│   ├── wishlist/          # Danh sách yêu thích
│   ├── order/             # Lịch sử đơn hàng
│   ├── authentication/    # Đăng nhập/Đăng ký
│   └── admin/             # Quản trị viên
└── shared/                # Assets, CSS chung
\end{verbatim}

\subsubsection*{Các chức năng chính}

\begin{itemize}
    \item \textbf{Trang chủ (Home)}: Hiển thị sản phẩm nổi bật với infinite scroll, tích hợp section ``Gợi ý dành cho bạn'' từ VieComRec API.
    \item \textbf{Trang sản phẩm (Products)}: Hiển thị danh sách sản phẩm theo phân trang (20 sản phẩm/trang), hỗ trợ lọc theo danh mục và tìm kiếm semantic bằng AI.
    \item \textbf{Chi tiết sản phẩm (ProductDetail)}: Hiển thị thông tin chi tiết, đánh giá, sản phẩm tương tự, hỗ trợ thêm giỏ hàng và wishlist.
    \item \textbf{Giỏ hàng (Cart)}: Quản lý sản phẩm trong giỏ, lưu trữ theo user trong localStorage với key \texttt{cart\_\{userId\}}.
    \item \textbf{Thanh toán (Checkout)}: Tích hợp Stripe payment gateway.
    \item \textbf{Wishlist}: Yêu cầu đăng nhập, đồng bộ với MongoDB.
    \item \textbf{Admin Panel}: Dashboard quản lý sản phẩm, đơn hàng, vận chuyển và AI Dashboard cho VieComRec.
\end{itemize}

\subsubsection*{VieComRec API Client}

File \texttt{client/src/api/viecomrec.js} cung cấp các hàm gọi API gợi ý:

\begin{verbatim}
// Gợi ý sản phẩm cho user
export const getRecommendations = async (userId, topk, excludeSeen);

// Tìm kiếm semantic tiếng Việt
export const semanticSearch = async (query, topk, filters);

// Sản phẩm tương tự (CF-based)
export const getSimilarItems = async (productId, topk);

// Tìm kiếm theo lịch sử mua hàng
export const getProfileBasedSearch = async (productHistory, topk);

// Scheduler APIs
export const getSchedulerStatus = async ();
export const triggerTraining = async (modelType);
export const getDriftStatus = async ();
\end{verbatim}

\subsubsection*{Luồng xử lý dữ liệu phía Client}

\begin{enumerate}
    \item Người dùng truy cập trang (Home/Products).
    \item Component gọi \texttt{loadRecommendations()} với \texttt{user\_id} từ localStorage.
    \item VieComRec API trả về danh sách \texttt{product\_id} với \texttt{score}.
    \item Client gọi \texttt{POST /api/products/arr} để lấy thông tin đầy đủ từ MongoDB.
    \item Merge dữ liệu VieComRec + MongoDB và render ProductCard.
\end{enumerate}

\subsection{Phân hệ Backend (Node.js/Express)}

Đây là lớp trung gian chịu trách nhiệm xử lý nghiệp vụ, xác thực và giao tiếp với các dịch vụ khác.

\subsubsection*{Công nghệ sử dụng}

\begin{itemize}
    \item \textbf{Node.js 16.x}: Runtime JavaScript server-side.
    \item \textbf{Express 4.16.1}: Web framework RESTful API.
    \item \textbf{Mongoose 6.3.3}: ODM cho MongoDB.
    \item \textbf{JWT (jsonwebtoken 8.5.1)}: Xác thực người dùng với token.
    \item \textbf{bcrypt 5.0.1}: Mã hoá mật khẩu.
    \item \textbf{Stripe 9.6.0}: Payment gateway integration.
    \item \textbf{Axios 0.27.2}: HTTP client gọi VieComRec API.
\end{itemize}

\subsubsection*{Cấu trúc API Routes}

\begin{verbatim}
server/
├── index.js               # Entry point, Express app
├── routes/
│   ├── auth.js            # /api/auth - Authentication
│   ├── products.js        # /api/products - Product CRUD
│   ├── orders.js          # /api/orders - Order management
│   ├── payments.js        # /api/payments - Stripe payments
│   ├── shipping.js        # /api/shipping - Shipment tracking
│   ├── recommend.js       # /api/recommend - VieComRec proxy
│   ├── ingest.js          # /api/ingest - ML data ingestion
│   └── notifications.js   # /api/notifications - Email
├── controller/            # Business logic
├── model/                 # Mongoose schemas
├── middleware/
│   └── auth.js            # JWT verification middleware
├── services/
│   └── BaseURLs.js        # Service URLs configuration
└── utils/                 # Pagination, ID generation
\end{verbatim}

\subsubsection*{Chi tiết các API Endpoints}

\textbf{Authentication API (/api/auth):}
\begin{itemize}
    \item \texttt{POST /register}: Đăng ký với bcrypt hash password.
    \item \texttt{POST /login}: Đăng nhập, trả về JWT token.
    \item \texttt{POST /verify}: Xác thực token hiện tại.
    \item \texttt{POST /role}: Kiểm tra quyền ADMIN.
    \item \texttt{POST /wishlist}: Lấy danh sách yêu thích.
    \item \texttt{PATCH /wishlist}: Thêm/xóa sản phẩm khỏi wishlist.
\end{itemize}

\textbf{Products API (/api/products):}
\begin{itemize}
    \item \texttt{GET /}: Lấy sản phẩm theo trang (20/page), hỗ trợ filter \texttt{category}.
    \item \texttt{GET /recommendations}: Random 2 categories với mỗi category 5 sản phẩm.
    \item \texttt{GET /:id}: Chi tiết sản phẩm theo \texttt{product\_id}.
    \item \texttt{GET /:id/reviews}: Đánh giá sản phẩm với pagination và sorting.
    \item \texttt{GET /:id/similar}: Sản phẩm tương tự theo category/brand/type.
    \item \texttt{POST /cart}: Validate giỏ hàng trước thanh toán.
    \item \texttt{POST /arr}: Lấy nhiều sản phẩm theo array \texttt{product\_id}.
    \item \texttt{PATCH /updateQuantity}: Cập nhật stock sau khi mua.
\end{itemize}

\textbf{Orders API (/api/orders):}
\begin{itemize}
    \item \texttt{POST /}: Tạo đơn hàng mới, tự động gọi Products, Shipping, Notifications.
    \item \texttt{GET /:id}: Chi tiết đơn hàng.
    \item \texttt{GET /}: Danh sách đơn hàng (Admin only).
    \item \texttt{PATCH /:id}: Cập nhật trạng thái (CREATED/PROCESSING/FULFILLED/CANCELLED).
\end{itemize}

\textbf{Recommendation Proxy (/api/recommend):}
\begin{itemize}
    \item \texttt{POST /}: Proxy đến VieComRec \texttt{/recommend}.
    \item \texttt{POST /search}: Proxy đến VieComRec \texttt{/search} (semantic search).
    \item \texttt{POST /similar}: Proxy đến VieComRec \texttt{/similar\_items}.
    \item \texttt{GET /health}: Kiểm tra trạng thái VieComRec service.
\end{itemize}

\textbf{Ingest API (/api/ingest) - Gửi dữ liệu đến ML:}
\begin{itemize}
    \item \texttt{POST /purchase}: Gửi thông tin mua hàng để cập nhật CF model.
    \item \texttt{POST /review}: Gửi đánh giá để cập nhật content-based model.
    \item \texttt{POST /batch}: Batch ingest nhiều purchases và reviews.
    \item \texttt{GET /stats}: Thống kê ingestion (Admin only).
\end{itemize}

\subsubsection*{Xử lý Fallback khi VieComRec không khả dụng}

Backend implement graceful degradation khi VieComRec API không phản hồi:

\begin{verbatim}
// server/routes/recommend.js
router.post("/", async (req, res) => {
  try {
    const response = await axios.post(
      `${VIECOMREC_BASEURL}/recommend`, {...}
    );
    res.json(response.data);
  } catch (error) {
    // Fallback to mock data if VieComRec is unavailable
    const mockRecommendations = [
      { rank: 1, product_id: 101, score: 0.91, ... },
      ...
    ];
    res.json({
      recommendations: mockRecommendations,
      is_fallback: true,
      fallback_method: "mock_data"
    });
  }
});
\end{verbatim}

\subsection{Phân hệ Recommender Service (VieComRec)}

VieComRec là dịch vụ gợi ý mỹ phẩm được triển khai bằng FastAPI và Docker Compose. Hệ thống hỗ trợ:

\begin{itemize}
    \item \textbf{Collaborative Filtering (ALS/BPR)}: Gợi ý dựa trên hành vi người dùng tương tự.
    \item \textbf{Content-based (PhoBERT embeddings)}: Gợi ý dựa trên nội dung sản phẩm với semantic search tiếng Việt.
    \item \textbf{Hybrid reranking}: Kết hợp CF score + content score + popularity để tối ưu độ chính xác.
\end{itemize}

\subsubsection*{API Endpoints}

Chi tiết về các API endpoints, SLA targets, và luồng xử lý được trình bày đầy đủ tại \textbf{Chương 5 - Kiến trúc Serving}.
Các endpoint chính bao gồm: \texttt{/recommend}, \texttt{/search}, \texttt{/similar\_items}, 
\texttt{/scheduler/status}, và \texttt{/health}.

Hệ thống hỗ trợ tối ưu hoá cho tình huống \textbf{cold-start user} (người dùng mới) bằng cách fallback sang content-based + popularity ranking khi user không có đủ lịch sử tương tác.

\subsection{Phân hệ Database (MongoDB)}

MongoDB 6.0 được triển khai qua Docker với cấu hình:

\begin{verbatim}
# docker-compose.yml
services:
  mongodb:
    image: mongo:6.0
    container_name: cosmetic_mongodb
    ports:
      - "27017:27017"
    environment:
      MONGO_INITDB_ROOT_USERNAME: admin
      MONGO_INITDB_ROOT_PASSWORD: password123
      MONGO_INITDB_DATABASE: cosmetic_db
    volumes:
      - mongodb_data:/data/db
      - ./mongo-init.js:/docker-entrypoint-initdb.d/mongo-init.js:ro
\end{verbatim}

MongoDB quản lý toàn bộ dữ liệu phi cấu trúc của hệ thống, đặc biệt phù hợp cho thương mại điện tử vì:

\begin{itemize}
    \item Cấu trúc linh hoạt (Document-based) phù hợp với dữ liệu sản phẩm đa dạng.
    \item Hỗ trợ scale-out với replica sets.
    \item Tương thích hoàn hảo với Node.js thông qua Mongoose ODM.
    \item Hỗ trợ text search index cho tìm kiếm sản phẩm.
\end{itemize}

\subsection{Luồng tương tác tổng thể}

\begin{figure}[H]
    \centering
    \includegraphics[width=0.85\textwidth]{image/data_flow_diagram.png}
    \caption{Luồng tương tác hệ thống}
\end{figure}

\begin{enumerate}
    \item Người dùng truy cập trang sản phẩm hoặc tìm kiếm.
    \item React Client gửi request đến Express Backend.
    \item Backend xử lý:
    \begin{itemize}
        \item Nếu cần data sản phẩm: truy vấn MongoDB.
        \item Nếu cần gợi ý: gọi VieComRec API với \texttt{user\_id}.
    \end{itemize}
    \item VieComRec truy xuất model (ALS/PhoBERT), tính toán và trả về danh sách \texttt{product\_id} với \texttt{score}.
    \item Backend truy vấn MongoDB để enrich thông tin sản phẩm (image, price, stock).
    \item Backend merge dữ liệu và trả về JSON response cho Frontend.
    \item Frontend render ProductCard với thông tin đầy đủ.
\end{enumerate}

\vspace{1em}

\section{Thiết kế xử lý dữ liệu}

\subsection{Nguồn dữ liệu}

Dữ liệu của hệ thống bao gồm:

\begin{itemize}
    \item \textbf{Dữ liệu sản phẩm}: Hơn 40 trường thông tin bao gồm tên, thương hiệu, danh mục, giá, ảnh, mô tả, thành phần, rating breakdown (1-5 sao), số lượng bán.
    \item \textbf{Dữ liệu hành vi người dùng}: Xem sản phẩm, thêm giỏ hàng, thêm wishlist, mua hàng với timestamp.
    \item \textbf{Dữ liệu đánh giá (Reviews)}: Rating, comment, product quality, variation đã mua.
    \item \textbf{Dữ liệu đơn hàng}: Lịch sử mua với trạng thái, địa chỉ giao hàng, timestamp.
\end{itemize}

Các dữ liệu này được lưu trong MongoDB và được VieComRec trích xuất thông qua Ingest API trong quá trình chạy batch offline (training).

\subsection{Tổ chức dữ liệu trong MongoDB}

\subsubsection*{Collection Users}

\begin{verbatim}
// server/model/Users.js
const UsersSchema = new Schema({
    first_name: { type: String },
    last_name: { type: String },
    email: { type: String, required: true },      // Unique key
    password: { type: String, required: true },   // bcrypt hashed
    role: { 
        type: String, 
        enum: ['USER', 'ADMIN'], 
        default: "USER" 
    },
    phone: { type: String },
    wishlist: { type: Array }  // Array of product_id
});
\end{verbatim}

\subsubsection*{Collection Products}

\begin{verbatim}
// server/model/Products.js
const productSchema = new Schema({
    product_id: { type: Number, unique: true, required: true },
    shop_id: { type: Number, default: 0 },
    name: String,
    product_name: String,
    brand: String,
    price: { type: Number, default: 0 },
    
    // Rating & Sales statistics
    avg_rating: { type: Number, default: 0 },
    avg_star: { type: Number, default: 0 },
    num_sold: { type: Number, default: 0 },
    num_rating: { type: Number, default: 0 },
    
    // Rating breakdown (for display)
    is_5_star: { type: Number, default: 0 },
    is_4_star: { type: Number, default: 0 },
    is_3_star: { type: Number, default: 0 },
    is_2_star: { type: Number, default: 0 },
    is_1_star: { type: Number, default: 0 },
    
    // Product details
    category: String,
    type: String,
    skin_kind: String,
    skin_type: String,
    origin: String,
    capacity: String,
    
    // Content for semantic search
    description: String,
    processed_description: String,  // Preprocessed for PhoBERT
    ingredient: String,
    feature: String,
    
    // Image path
    image: String,
    stock: { type: Number, default: 100 }
}, { timestamps: true });

// Text search index
productSchema.index({ 
    name: 'text', 
    product_name: 'text', 
    brand: 'text', 
    description: 'text' 
});
\end{verbatim}

\subsubsection*{Collection Orders}

\begin{verbatim}
// server/model/Orders.js
const orderSchema = mongoose.Schema({
    order_id: { type: String, required: true, unique: true },
    user_id: { 
        type: mongoose.Schema.Types.ObjectId, 
        ref: 'Users' 
    },
    name: { first: String, last: String },
    email: { type: String },
    phone_number: { type: String },
    address: {
        country: String,
        city: String,
        area: String,
        street: String,
        building_number: String,
        floor: String,
        apartment_number: String
    },
    ordered_at: { type: Date, default: Date.now },
    status: {
        type: String,
        enum: ['CREATED', 'PROCESSING', 'FULFILLED', 'CANCELLED'],
        default: 'CREATED'
    },
    products: { type: Array, required: true, default: [] },
    total: { type: Number, required: true },
    
    // ML tracking
    ingested: { type: Boolean, default: false },
    ingest_timestamp: { type: Date }
});
\end{verbatim}

\subsubsection*{Collection Reviews}

\begin{verbatim}
// server/model/Reviews.js
const reviewSchema = new Schema({
    review_id: { type: Number },
    user_id: { type: Number, required: true, index: true },
    product_id: { type: Number, required: true, index: true },
    rating: { type: Number, required: true, min: 1, max: 5 },
    product_quality: { type: Number, min: 1, max: 5 },
    
    comment: String,
    processed_comment: String,  // Preprocessed for sentiment
    
    product_name: String,
    variation: String,
    cmt_date: { type: Date },
    created_at: { type: Date, default: Date.now }
}, { timestamps: true });

// Compound indexes for efficient queries
reviewSchema.index({ user_id: 1, product_id: 1 });
reviewSchema.index({ product_id: 1, rating: -1 });
\end{verbatim}

\vspace{1em}

\section{Tích hợp VieComRec vào hệ thống}

\subsection{Bước 1: Cấu hình kết nối}

Cấu hình Base URL trong cả Backend và Frontend:

\begin{verbatim}
// server/services/BaseURLs.js
export const VIECOMREC_BASEURL = process.env.VIECOMREC_API 
    || 'http://localhost:8000';

// client/src/api/BaseURLs.js
export const VIECOMREC_BASEURL = 
    process.env.REACT_APP_VIECOMREC_URL || "http://localhost:8000";
\end{verbatim}

\subsection{Bước 2: Khởi chạy VieComRec bằng Docker}

\begin{verbatim}
# Build và chạy VieComRec container
docker-compose build
docker-compose up -d

# Kiểm tra health
curl http://localhost:8000/health
\end{verbatim}

API mặc định chạy tại: \texttt{http://localhost:8000}

\subsection{Bước 3: Backend Proxy Implementation}

\begin{verbatim}
// server/routes/recommend.js
router.post("/", async (req, res) => {
  const { user_id, topk = 10, exclude_seen = true, 
          filter_params = null } = req.body;

  try {
    const response = await axios.post(
      `${VIECOMREC_BASEURL}/recommend`, 
      { user_id, topk, exclude_seen, filter_params, rerank: true }
    );
    res.json(response.data);
  } catch (error) {
    // Fallback handling...
  }
});
\end{verbatim}

\subsection{Bước 4: Frontend gọi và hiển thị}

\begin{verbatim}
// client/src/pages/home/Home.js
const loadRecommendations = async () => {
  const profile = JSON.parse(localStorage.getItem("profile"));
  const userId = profile?.user?.user_id || 1;

  // 1. Gọi VieComRec API
  const result = await getRecommendations(userId, 10, true);

  if (result && result.recommendations) {
    // 2. Lấy product_ids
    const productIds = result.recommendations.map(
      item => item.product_id
    );
    
    // 3. Enrich từ MongoDB
    const productsRes = await axios.post(
      `${PRODUCTS_BASEURL}/arr`, { arr: productIds }
    );
    
    // 4. Merge data
    const mapped = result.recommendations.map((item) => {
      const dbProduct = productMap[item.product_id] || {};
      return {
        product_id: item.product_id,
        name: dbProduct.name || item.product_name,
        price: dbProduct.price || item.price,
        image: dbProduct.image,
        score: item.score,
        ...
      };
    });
    setRecommendations(mapped);
  }
};
\end{verbatim}

\subsection{Bước 5: Ingest dữ liệu mới cho ML}

Khi user mua hàng hoặc đánh giá, Backend gửi dữ liệu đến VieComRec:

\begin{verbatim}
// server/controller/me/Ingest.js
export const ingestPurchase = async (req, res) => {
  const { user_id, product_id, quantity, order_id } = req.body;

  // Send to VieComRec API
  await axios.post(`${VIECOMREC_API}/ingest/purchase`, {
    user_id,
    product_id: parseInt(product_id),
    quantity: parseInt(quantity) || 1,
    timestamp: new Date().toISOString()
  });

  // Mark order as ingested
  await Order.findOneAndUpdate(
    { order_id },
    { ingested: true, ingest_timestamp: new Date() }
  );
};
\end{verbatim}

\vspace{1em}

\section{Sơ đồ luồng dữ liệu chi tiết}

\subsection{Luồng gợi ý sản phẩm (Recommendation Flow)}

\begin{enumerate}
    \item User mở trang Home hoặc Products.
    \item React gọi \texttt{getRecommendations(userId, 10, true)}.
    \item VieComRec kiểm tra:
    \begin{itemize}
        \item Nếu user có $\geq 2$ lịch sử mua hàng $\rightarrow$ sử dụng Collaborative Filtering (ALS).
        \item Nếu user mới (cold-start) $\rightarrow$ fallback content-based + popularity ranking.
    \end{itemize}
    \item VieComRec thực hiện Hybrid Reranking: \\
    $final\_score = \alpha \cdot CF\_score + \beta \cdot content\_score + \gamma \cdot popularity$
    \item Trả về danh sách \texttt{[\{product\_id, score, rank\}]}.
    \item Backend/Frontend gọi \texttt{POST /api/products/arr} để lấy thông tin từ MongoDB.
    \item Frontend hiển thị section ``Gợi ý dành cho bạn''.
\end{enumerate}

\subsection{Luồng tìm kiếm ngữ nghĩa (Semantic Search Flow)}

\begin{enumerate}
    \item User nhập query tìm kiếm (ví dụ: ``serum vitamin c cho da dầu'').
    \item React gọi \texttt{semanticSearch(query, 20)}.
    \item VieComRec:
    \begin{itemize}
        \item Encode query bằng PhoBERT $\rightarrow$ query\_embedding.
        \item Tính cosine similarity với product\_embeddings đã index.
        \item Rerank theo relevance + popularity.
    \end{itemize}
    \item Trả về danh sách với \texttt{semantic\_score} và \texttt{final\_score}.
    \item Frontend hiển thị với badge ``Tìm kiếm thông minh bằng AI''.
\end{enumerate}

\subsection{Luồng mua hàng và cập nhật ML}

\begin{enumerate}
    \item User thêm sản phẩm vào giỏ và thanh toán.
    \item Backend tạo Order, gọi Stripe, cập nhật stock.
    \item Backend gọi \texttt{POST /api/ingest/purchase} với dữ liệu đơn hàng.
    \item VieComRec nhận và lưu interaction mới.
    \item Scheduler tự động retrain model định kỳ (hoặc khi detect drift).
\end{enumerate}

\vspace{1em}

\section{Tổng kết}

Hệ thống RabbitMart + VieComRec được thiết kế với các đặc điểm nổi bật:

\begin{itemize}
    \item \textbf{Kiến trúc Microservices}: Tách biệt Frontend, Backend, Database và ML Service để dễ dàng scale và maintain.
    \item \textbf{RESTful API chuẩn}: Giao tiếp thống nhất giữa các services qua HTTP/JSON.
    \item \textbf{Graceful Degradation}: Hệ thống vẫn hoạt động khi VieComRec không khả dụng nhờ fallback logic.
    \item \textbf{Real-time Data Ingestion}: Dữ liệu mua hàng và đánh giá được gửi ngay đến ML để cập nhật model.
    \item \textbf{Cold-start Handling}: Người dùng mới vẫn nhận được gợi ý qua content-based + popularity.
    \item \textbf{Semantic Search tiếng Việt}: PhoBERT cho phép tìm kiếm theo ngữ nghĩa, hiểu context query.
    \item \textbf{Admin Dashboard}: Quản lý toàn diện sản phẩm, đơn hàng và monitoring ML model.
\end{itemize}
