% --- NỘI DUNG CHO FILE: detail/thanks.tex ---
% Đã thêm \parskip để tạo khoảng cách giữa các đoạn

\pagestyle{plain} % Hiển thị số trang

\begin{center}
    {\LARGE \textbf{LỜI CẢM ƠN}} \\
    \vspace{1.5cm} % Khoảng cách cố định giữa tiêu đề và nội dung
\end{center}
    
% --- Bắt đầu khối định dạng đặc biệt cho trang này ---
\begingroup % Sử dụng group để định dạng không ảnh hưởng các trang sau

% 1. Đặt cỡ chữ 13pt và line spacing cơ bản (16pt)
\fontsize{13pt}{16pt}\selectfont 

% 2. Đặt khoảng cách dòng (giả sử là 1.3)
\begin{spacing}{1.3} 

% 3. Đặt thụt đầu dòng
\setlength{\parindent}{1cm} 

% 4. THÊM KHOẢNG CÁCH GIỮA CÁC ĐOẠN
% (Bạn có thể thay 6pt bằng 8pt hoặc 10pt nếu muốn)
\setlength{\parskip}{6pt} 

% 5. NỘI DUNG VĂN BẢN

% Đoạn này sẽ thụt lề

Trong suốt quá trình hoàn thành báo cáo đồ án môn học \textit{Nhập môn Trí tuệ Nhân tạo} với đề tài ``Gợi ý sản phẩm sữa rửa mặt'', nhóm chúng em đã nhận được sự hỗ trợ và hướng dẫn quý báu từ nhiều phía.

Trước hết, nhóm xin gửi lời biết ơn sâu sắc đến thầy \textbf{Hoàng Anh Đức} -- người đã trực tiếp hướng dẫn, tận tình chỉ bảo và đưa ra những góp ý chuyên môn quan trọng, giúp nhóm hoàn thiện đề tài một cách tốt nhất.

Nhóm cũng xin chân thành cảm ơn các thầy/cô giảng dạy môn \textit{Nhập môn Trí tuệ Nhân tạo} đã trang bị cho chúng em những kiến thức nền tảng về trí tuệ nhân tạo, và các kiến thức liên quan, đã tạo tiền đề quan trọng để nhóm thực hiện đề tài này.

Bên cạnh đó, chúng em xin cảm ơn các thành viên trong nhóm đã luôn hợp tác, trao đổi ý tưởng và hỗ trợ lẫn nhau trong suốt quá trình làm việc.

Mặc dù nhóm đã cố gắng hết sức, nhưng khó tránh khỏi những thiếu sót. Chúng em rất mong nhận được sự góp ý từ thầy/cô và các bạn để báo cáo được hoàn thiện hơn.


% Đoạn này sẽ thụt lề và cách đoạn trên 6pt
Nhóm chúng em xin chân thành cảm ơn!

\end{spacing} % Kết thúc môi trường dãn dòng
\endgroup   % Kết thúc nhóm định dạng
% --- Hết khối định dạng ---