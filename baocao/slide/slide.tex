\documentclass{beamer}

\usetheme{Madrid} % bạn có thể đổi theme: Berlin, CambridgeUS, Warsaw,...

\usepackage[utf8]{inputenc}
\usepackage[vietnamese]{babel}
\usepackage{lipsum}

% Title Information
\title[Hệ thống gợi ý mỹ phẩm Hybrid]{BÁO CÁO CUỐI KÌ\\Hệ thống Gợi ý Mỹ phẩm Hybrid (ALS + BPR + PhoBERT)}
\author{Nhóm thực hiện: Nhóm 21}
\institute{Trường Đại học Khoa học Tự nhiên - ĐHQGHN}
\date{\today}

\begin{document}

%-------------------------------------------------------
\begin{frame}
    \titlepage
\end{frame}

%---------------------------------------------------

%=======================================================
%======================
%--------------------------------------------------
\begin{frame}[plain]
    \centering
    {\Huge \textbf{1. GIỚI THIỆU}}
\end{frame}
%--------------------------------------------------

\section{Giới thiệu}
%======================

%--------------------------------------------------
\begin{frame}{Đặt vấn đề}
\textbf{Bối cảnh ngành thương mại điện tử Việt Nam:}
\begin{itemize}
    \item Thị trường TMĐT Việt Nam 2023: 16.4 tỷ USD, CAGR > 20\%.
    \item Ngành mỹ phẩm - làm đẹp: tăng trưởng nhanh, cạnh tranh cao.
    \item Người dùng yêu cầu \textbf{cá nhân hóa trải nghiệm mua sắm} (91\% chọn đề xuất phù hợp nhu cầu cá nhân).
\end{itemize}

\textbf{Thách thức dữ liệu mỹ phẩm Việt Nam:}
\begin{itemize}
    \item \textbf{Dữ liệu thưa (Sparsity)}: ~99.9\% interactions bị thiếu.
    \item \textbf{Rating lệch (Skewed Ratings)}: >85\% đánh giá 5 sao.
    \item \textbf{Cold-start users}: 30-40\% người dùng hàng tháng là mới.
\end{itemize}
\end{frame}
%--------------------------------------------------

%--------------------------------------------------
\begin{frame}{Mục tiêu đề tài}
\textbf{Mục tiêu tổng quát:} 
Xây dựng hệ thống gợi ý lai (Hybrid Recommender) cho mỹ phẩm, giải quyết sparsity, rating skew và cold-start, nâng cao cá nhân hóa.

\textbf{Mục tiêu cụ thể:}
\begin{enumerate}
    \item \textbf{Hệ thống Hybrid CF + Content-based:} 
    \begin{itemize}
        \item CF: ALS với implicit feedback, regularization mạnh.
        \item Content-based: PhoBERT embeddings, similarity trong không gian ngữ nghĩa.
        \item Kết hợp tuyến tính với cơ chế chuyển đổi CF / Content-based.
    \end{itemize}
    \item \textbf{Ứng dụng Deep Learning (PhoBERT) cho ngôn ngữ Việt:} 
    \begin{itemize}
        \item Tiền xử lý text: chuẩn hóa Unicode, tách từ, chuẩn hóa tên thành phần.
        \item Trích embedding 768-dim, tính cosine similarity, xây dựng knowledge graph.
    \end{itemize}
    \item \textbf{Triển khai theo hướng MLOps:} 
    \begin{itemize}
        \item Pipeline tự động: thu thập, train, validation, tuning, CI/CD.
        \item Monitoring: data drift, metrics (precision, recall, NDCG, CTR).
        \item Model registry, container hóa, Kubernetes autoscaling.
    \end{itemize}
\end{enumerate}
\end{frame}
%--------------------------------------------------

%--------------------------------------------------

\begin{frame}[plain]
    \centering
    {\Huge \textbf{2. CHIẾN LƯỢC NGƯỜI DÙNG VÀ PHÂN KHÚC}}
\end{frame}

%--------------------------------------------------
\begin{frame}{2.2 Chuẩn hóa Tiếng Việt và Sửa lỗi Chính tả}
\textbf{Thách thức dữ liệu:} teencode, lỗi gõ nhanh, từ dính, token sai, emoji mã hoá.

\textbf{Quy trình Hybrid AI--Human gồm 6 giai đoạn:}
\begin{enumerate}
    \item Trích xuất từ vựng (50k+ token).
    \item Tiền lọc: bỏ từ sạch, thương hiệu, rác, từ có \_.
    \item Gọi AI (Gemini 2.5 Flash) theo \textbf{chunk 200 từ}.
    \item Kiểm định con người: top 10k từ xuất hiện nhiều.
    \item Hậu xử lý: fix từ ghép tách sai (``đư ợc'' → ``được'').
    \item Cập nhật từ điển viết tắt thủ công.
\end{enumerate}

\textbf{Hiệu quả:} sửa đúng 94.2\%, sai 3.1\%, tiết kiệm 99\% chi phí API.
\end{frame}
%--------------------------------------------------


%--------------------------------------------------
\begin{frame}{2.3 Phân khúc Người dùng (User Segmentation)}
\begin{itemize}
    \item \textbf{Dữ liệu tương tác:}
    \begin{itemize}
        \item Chỉ \(\approx 8.6\%\) người dùng có \(\ge 2\) tương tác.
    \end{itemize}

    \item \textbf{Chiến lược định tuyến theo loại user:}
    \begin{itemize}
        \item \textbf{Trainable Users:}
        \begin{itemize}
            \item Collaborative Filtering (ALS/BPR).
            \item Reranking theo chất lượng bình luận.
        \end{itemize}

        \item \textbf{Cold-start Users:}
        \begin{itemize}
            \item Content-based: PhoBERT embedding similarity.
            \item Popularity-based fallback.
        \end{itemize}
    \end{itemize}
\end{itemize}
\end{frame}
%--------------------------------------------------


%--------------------------------------------------
\begin{frame}{2.4 Tích hợp BERT Embeddings}
\begin{itemize}
    \item \textbf{Input kết hợp:}
    \begin{itemize}
        \item Tên sản phẩm
        \item Công dụng
        \item Thành phần
        \item Loại da phù hợp
    \end{itemize}

    \item \textbf{Output:} Vector ngữ nghĩa (BERT embedding).
    \begin{itemize}
        \item Dùng tính độ tương đồng nội dung.
        \item Dùng làm \textbf{khởi tạo tham số} cho ALS → giúp hội tụ nhanh và chính xác hơn.
    \end{itemize}
\end{itemize}
\end{frame}
%--------------------------------------------------

%--------------------------------------------------
\begin{frame}[plain]
    \centering
    {\Huge \textbf{3. HUẤN LUYỆN MÔ HÌNH}}
\end{frame}
%--------------------------------------------------

%=======================================================
%===========================================
\section{Huấn luyện mô hình (Implementation Details)}
%===========================================

%-------------------------------------------
\begin{frame}{3.1 Alternating Least Squares (ALS)}
%-------------------------------------------
\textbf{ALS} là thuật toán Matrix Factorization tối ưu cho dữ liệu implicit.

\begin{itemize}
    \item \textbf{Đầu vào}: Ma trận Confidence dạng CSR (rất thưa, density 0.11\%).
    \item Confidence score:
    \[
        c_{ui} = r_{ui} + q_{ui}
    \]
    \item Giá trị $\; c_{ui} \in [1, 6]$: phân biệt rating 5 sao thật vs. 5 sao nhiễu.
    \item Kích thước: $26{,}000 \times 2{,}200$, nnz $\approx 65{,}000$.
    \item Thách thức: dữ liệu cực thưa $\Rightarrow$ dễ trôi dạt embedding.
\end{itemize}
\end{frame}

%-------------------------------------------
\begin{frame}{ALS – BERT Initialization (Đóng góp chính)}
%-------------------------------------------
\textbf{Vấn đề}: Khởi tạo Gaussian ngẫu nhiên khiến cold items bị drift.

\textbf{Giải pháp}: \textbf{BERT Initialization}.

\begin{enumerate}
    \item Trích xuất embedding của sản phẩm bằng \textbf{PhoBERT}:
    \[
        e_i = \text{PhoBERT}(t_i)_{\text{[CLS]}} \in \mathbb{R}^{768}
    \]
    \item Giảm chiều bằng \textbf{TruncatedSVD} xuống $d=64$.
    \item Căn chỉnh theo \texttt{item\_to\_idx}.
    \item Gán làm item factors khởi tạo:
    \[
        V^{(0)} = \text{AlignedBERTEmbeddings}
    \]
\end{enumerate}

\textbf{Lợi ích:}
\begin{itemize}
    \item Hội tụ nhanh hơn (15 iterations).
    \item Cold items được neo ngữ nghĩa.
    \item Chuyển tri thức từ NLP $\rightarrow$ CF.
\end{itemize}

\textbf{Kết quả:} Recall@10 tăng \textbf{+33\%}.
\end{frame}

%-------------------------------------------
\begin{frame}{ALS – Hyperparameters \& Training Loop}
%-------------------------------------------
\begin{itemize}
    \item \textbf{factors = 64} \quad không gian tiềm ẩn
    \item \textbf{regularization = 0.1} \quad giữ ổn định BERT init
    \item \textbf{alpha = 10} \quad phù hợp confidence [1,6]
    \item \textbf{iterations = 15}
\end{itemize}

\textbf{Hàm mất mát ALS:}
\[
L = \sum_{u,i} C_{ui}(p_{ui} - u_u^\top v_i)^2 
+ \lambda \left( \sum_u \|u_u\|^2 + \sum_i \|v_i\|^2 \right)
\]

\textbf{Tối ưu luân phiên:}
\[
u_u = (V^\top C_u V + \lambda I)^{-1} V^\top C_u p_u
\]
\[
v_i = (U^\top C_i U + \lambda I)^{-1} U^\top C_i p_i
\]

Huấn luyện dùng \textbf{implicit C++ backend}, thời gian: \textbf{1–2 phút}.
\end{itemize}
\end{frame}

%-------------------------------------------
\begin{frame}{3.2 Bayesian Personalized Ranking (BPR)}
%-------------------------------------------
\textbf{Mục tiêu}: tối ưu \textbf{ranking} (pairwise), không dự đoán rating tuyệt đối.

\textbf{Đầu vào:} bộ ba (u, i, j)
\begin{itemize}
    \item $i$: positive item ($r_{ui} \ge 4$)
    \item $j$: negative item (chưa tương tác hoặc rating thấp)
\end{itemize}

\textbf{Số lượng triplets:}  
\[
|D_S| = 65{,}000 \times 5 = 325{,}000
\]

\textbf{Mục tiêu tối ưu:}
\[
\hat{r}_{ui} > \hat{r}_{uj}, \quad \forall (u,i,j) \in D_S
\]
\end{frame}

%-------------------------------------------
\begin{frame}{BPR – Hard Negative Mining}
%-------------------------------------------
\textbf{Vấn đề:} random negatives quá dễ ⇒ gradient gần 0.

\textbf{Giải pháp: Dual Hard Negative Mining}
\begin{enumerate}
    \item \textbf{Explicit hard negatives}: rating thấp ($\le 3$).
    \item \textbf{Implicit hard negatives}: sản phẩm phổ biến nhưng user không mua.
\end{enumerate}

\textbf{Mixing:}
\[
p(\text{hard}) = 0.3,\quad p(\text{random}) = 0.7
\]

\textbf{Lợi ích:}
\begin{itemize}
    \item Gradient giàu thông tin hơn.
    \item Giảm popularity bias.
    \item Học được ranh giới tinh tế giữa các sản phẩm tương tự.
\end{itemize}
\end{frame}

%-------------------------------------------
\begin{frame}{BPR – Loss Function}
%-------------------------------------------
\textbf{Hàm mất mát BPR:}
\[
L_{\text{BPR}} = -\sum_{(u,i,j)} \ln \sigma(\hat{r}_{ui} - \hat{r}_{uj})
+ \lambda (\|u_u\|^2 + \|v_i\|^2 + \|v_j\|^2)
\]

\textbf{Huấn luyện:}
\begin{itemize}
    \item SGD + mini-batch
    \item Xavier init cho embeddings
    \item Adam optimizer
    \item 50–80 epochs
\end{itemize}

\textbf{Kết quả:} cải thiện ranking cho các user có hành vi đa dạng.
\end{frame}

%-------------------------------------------
\begin{frame}{3.3 Model Registry \& Versioning}
%-------------------------------------------
Tất cả artifacts được lưu trữ theo version:

\texttt{artifacts/cf/als/v1\_20251127/}
\begin{itemize}
    \item \textbf{als\_U.npy} — user factors $(26000, 64)$
    \item \textbf{als\_V.npy} — item factors $(2200, 64)$
    \item \textbf{als\_params.json} — hyperparameters
    \item \textbf{als\_metrics.json} — Recall@10, NDCG@10
    \item \textbf{als\_metadata.json} — score ranges, BERT init info, git commit
\end{itemize}

\textbf{Mục đích:}
\begin{itemize}
    \item Tái lập mô hình (reproducibility)
    \item So sánh version
    \item Hỗ trợ Hybrid Reranking (chuẩn hóa CF scores)
\end{itemize}
\end{frame}


%--------------------------------------------------
\begin{frame}[plain]
    \centering
    {\Huge \textbf{4. HỆ THỐNG SERVICE VÀ HYBRID RERANKING}}
\end{frame}
%--------------------------------------------------


%=======================================================
\section{Serving và Hybrid Reranking}
%=======================================================

%=====================================================
\section{Hệ thống Serving và Hybrid Reranking}
%=====================================================

%-----------------------------------------------------
\begin{frame}{4.1 Kiến trúc Serving}
\textbf{Yêu cầu hệ thống:}
\begin{itemize}
    \item Độ trễ P95 $<$ 100ms
    \item Availability $\ge 99.9\%$
    \item Throughput $\ge 100$ req/s
\end{itemize}

\textbf{Luồng xử lý (Latency Breakdown):}
\[
\text{Client} 
\rightarrow \text{API Gateway (5ms)} 
\rightarrow \text{User Router (10ms)}
\rightarrow \text{Scoring Engine (50--80ms)}
\rightarrow \text{Response (5ms)}
\]

\textbf{Thành phần chính:}
\begin{itemize}
    \item API Gateway: validation, auth, rate limit.
    \item User Router: quyết định CF / Fallback.
    \item Scoring Engine: CF scoring hoặc Content scoring.
    \item Reranking: tích hợp multi-signal (nếu bật).
\end{itemize}
\end{frame}
%-----------------------------------------------------

%-----------------------------------------------------
\begin{frame}{Kiến trúc Phân Tầng (Layered Architecture)}
\begin{itemize}
    \item L1: \textbf{API Layer} – FastAPI entrypoint
    \item L2: \textbf{Business Logic} – Recommender, Fallback, Reranker, Cache
    \item L3: \textbf{Search Layer} – PhoBERT semantic search
    \item L4: \textbf{Configuration Layer}
\end{itemize}

\textbf{Nguyên tắc:}
\begin{itemize}
    \item Single Responsibility
    \item Dependency Injection
    \item Graceful degradation với multi-layer fallback
\end{itemize}
\end{frame}
%-----------------------------------------------------

%-----------------------------------------------------
\begin{frame}{Serving: Model Management \& Hot Reload}
\textbf{Model Registry:}
\begin{itemize}
    \item Quản lý nhiều version model (ALS/CF).
    \item Chứa score-range, metadata, đường dẫn artifacts.
\end{itemize}

\textbf{Hot-Reload Protocol:}
\begin{enumerate}
    \item So sánh model hiện tại với registry.
    \item Tải $U',V'$ mới.
    \item Validate kích thước, score-range.
    \item Atomic swap: $(U,V) \leftarrow (U',V')$.
\end{enumerate}

\textbf{Lợi ích:} Zero-downtime deployment.
\end{frame}
%-----------------------------------------------------

%-----------------------------------------------------
\begin{frame}{User Routing: Trainable vs Cold-start}
\textbf{Phân đoạn người dùng:}
\begin{itemize}
    \item Trainable users (8.6\%): $|H_u|\ge 2$ và có rating $\ge 4$.
    \item Cold-start users (91.4\%): còn lại.
\end{itemize}

\textbf{Routing rule:}
\[
path(u) = 
\begin{cases}
\text{CF Path}, & \text{nếu user trainable} \\
\text{Fallback Path}, & \text{ngược lại}
\end{cases}
\]

\textbf{Ý nghĩa:}
\begin{itemize}
    \item CF chỉ dùng cho user đủ dữ liệu.
    \item Cold-start ưu tiên Content + Popularity.
\end{itemize}
\end{frame}
%-----------------------------------------------------

%-----------------------------------------------------
\begin{frame}{CF Path: Scoring Pipeline}
\textbf{6 bước xử lý:}
\begin{enumerate}
    \item Index mapping user $\rightarrow u_{\text{cf}}$
    \item CF scoring: $ \hat{r}_{ui}=u^\top v_i $
    \item Seen-item filtering
    \item Attribute filtering
    \item Top-K (lấy 5K nếu có reranking)
    \item Hybrid Reranking
\end{enumerate}

\textbf{Độ phức tạp:}
\[
O(d \cdot |I|),\quad d=embedding\ dimension
\]
\end{frame}
%-----------------------------------------------------

%-----------------------------------------------------
\begin{frame}{Fallback Path: Content-based + Popularity}
\textbf{Score components:}
\[
s_{\text{content}}(u,i)=\cos(\tilde e_u,\tilde e_i)
\]
\[
s_{\text{pop}}(i)=\frac{\log(1+\text{sold}_i)}
{\max_j\log(1+\text{sold}_j)}
\]

\textbf{Hybrid score:}
\[
S_{\text{fallback}}(u,i)
=0.7\,s_{\text{content}}+0.3\,s_{\text{pop}}
\]

\textbf{New user ($m=0$):} chỉ dùng popularity.
\end{frame}
%-----------------------------------------------------

%-----------------------------------------------------
\begin{frame}{Performance Optimization \& Fallback}
\textbf{Caching:}
\begin{itemize}
    \item Popular items cache: top-50
    \item User profile cache (LRU)
    \item Similarity cache (2,500 pairs)
\end{itemize}

\textbf{Batch processing:}
\[
\hat R_{\text{batch}} = U_{\text{batch}}V^\top
\]

\textbf{Fallback layers:}
\begin{itemize}
    \item CF lỗi → Content-based
    \item PhoBERT lỗi → Popularity-only
    \item Filter xoá hết items → Bỏ filter
\end{itemize}
\end{frame}
%-----------------------------------------------------

%=====================================================
\section{Hybrid Reranking}
%=====================================================

%-----------------------------------------------------
\begin{frame}{Tổng quan Hybrid Reranking}
\textbf{Mục tiêu:}
\begin{itemize}
    \item Tối ưu relevance (độ phù hợp)
    \item Tăng diversity (đa dạng)
    \item Kết hợp nhiều tín hiệu scoring
\end{itemize}

\textbf{Input:} Candidate set $C_u$ từ CF hoặc Fallback.

\textbf{Output:} Top-$K$ danh sách cuối cùng.
\end{itemize}
\end{frame}
%-----------------------------------------------------

%-----------------------------------------------------
\begin{frame}{4 Tín hiệu chính (Normalized to [0,1])}
\begin{itemize}
    \item \textbf{1. CF score} – từ $u^\top v_i$
    \item \textbf{2. Content score} – cosine similarity
    \item \textbf{3. Popularity} – log-normalized
    \item \textbf{4. Quality signals} – rating, CTR, stability
\end{itemize}

\textbf{Hybrid score tổng quát:}
\[
S(u,i)=
w_{cf}s_{cf}+w_{content}s_{content}
+w_{pop}s_{pop}+w_qs_q
\]
\end{frame}
%-----------------------------------------------------

%-----------------------------------------------------
\begin{frame}{Diversity Re-ranking}
\textbf{Phương pháp: Maximal Marginal Relevance (MMR)}
\[
\text{MMR}(i)=
\lambda S(u,i)
-(1-\lambda)\max_{j\in R}\text{sim}(i,j)
\]

\textbf{Ý nghĩa:}
\begin{itemize}
    \item $\lambda$ cao → ưu tiên relevance
    \item $\lambda$ thấp → ưu tiên diversity
\end{itemize}
\end{frame}
%-----------------------------------------------------

%=====================================================
\section{Smart Search Integration}
%=====================================================

%-----------------------------------------------------
\begin{frame}{4.3 Smart Search Integration}
\textbf{Smart Search sử dụng:}
\begin{itemize}
    \item PhoBERT Embeddings
    \item Metadata + caching dùng chung với Serving
\end{itemize}

\textbf{Luồng tích hợp:}
\[
\text{CF/Fallback Output}
\rightarrow \text{Hybrid Reranking}
\rightarrow \text{Response}
\]
\[
\text{Smart Search Output}
\longrightarrow \text{Hybrid Reranking}
\]

\textbf{Lợi ích:}
\begin{itemize}
    \item Một pipeline thống nhất cho search + recommend
    \item Tối ưu reuse caching, model loading, monitoring
\end{itemize}
\end{frame}
%-----------------------------------------------------

%=======================================================
%======================

%--------------------------------------------------
\begin{frame}[plain]
    \centering
    {\Huge \textbf{5. ĐÁNH GIÁ VÀ THỰC NGHIỆM}}
\end{frame}
%--------------------------------------------------

\section{Đánh giá và Thực nghiệm}
%======================

%--------------------------------------------------
\begin{frame}{F.1 Kết quả Thực nghiệm – So sánh CF Models}
\begin{itemize}
    \item \textbf{BERT-ALS} đạt Recall@10 cao nhất: \textbf{0.1888} 
    \begin{itemize}
        \item +243.6\% so với Popularity baseline.
        \item Improving cả Recall và NDCG so với ALS.
    \end{itemize}

    \item \textbf{Tất cả mô hình CF} đều vượt trội hơn baseline (p < 0.05).

    \item \textbf{BPR}: coverage cao nhất (68.5\%), nhưng Recall thấp hơn ALS/BERT-ALS.

    \item \textbf{Cold-Aug variants}: Recall tốt nhưng NDCG thấp (~0.085) → ranking quality giảm.
\end{itemize}
\end{frame}
%--------------------------------------------------


%--------------------------------------------------
\begin{frame}{5.1 Kết quả Thực nghiệm – Cải thiện so với Baseline}
\textbf{Cải thiện Recall@10 so với Popularity Baseline:}

\begin{itemize}
    \item \textbf{BERT-ALS (best)}: +243.6\% (cao nhất)
    \item ALS checkpoint: +235.2\%
    \item ALS artifact: +232.6\%
    \item BPR advanced: +87.2\%
\end{itemize}

\textbf{Nhận xét:}
\begin{itemize}
    \item Kết quả vượt mục tiêu đề ra: Recall@10 \textgreater 0.20 (đạt 0.1888 ≈ mục tiêu).
    \item BERT initialization mang lại lợi thế mạnh khi dữ liệu sparse + domain-specific.
\end{itemize}
\end{frame}
%--------------------------------------------------


%--------------------------------------------------
\begin{frame}{So sánh ALS vs BERT-ALS}
\begin{itemize}
    \item Recall@10: tăng +2.5\% (0.1842 → 0.1888)
    \item NDCG@10: tăng +1.2\%
    \item Coverage: giảm mạnh (-63.4\%)
    \item Diversity: giảm (-55.3\%)
\end{itemize}

\textbf{Trade-off:}
\begin{itemize}
    \item Accuracy ↑ nhưng Coverage/Diversity ↓
    \item Nguyên nhân: BERT embeddings ưu tiên các item tương đồng về ngữ nghĩa
\end{itemize}
\end{frame}
%--------------------------------------------------


%--------------------------------------------------
\begin{frame}{ Hiệu quả Hybrid Reranking}
\textbf{Kết quả:}
\begin{itemize}
    \item \textbf{Recall@20 tăng}: +9.2\% → cải thiện long-tail.
    \item \textbf{Recall@10 giảm nhẹ}: -1.8\% (không có ý nghĩa thống kê).
    \item \textbf{Diversity giảm}: -2.2\%.
    \item \textbf{Coverage giảm}: -41.1\%.
    \item \textbf{Latency}: tăng từ 0.56ms → 2.72ms (vẫn < 10ms).
\end{itemize}

\textbf{Kết luận:}
\begin{itemize}
    \item Hybrid không cải thiện đáng kể so với CF-only vì CF đã tích hợp content signal từ BERT.
    \item Hybrid phù hợp cho: long-tail (K lớn), cold-start users.
\end{itemize}
\end{frame}
%--------------------------------------------------



%======================
\begin{frame}{F.2 Phân tích Chi tiết – Coverage vs Recall Trade-off}
\begin{itemize}
    \item \textbf{BERT-ALS}: Recall cao nhất nhưng Coverage thấp → phù hợp accuracy-first.
    \item \textbf{ALS}: Recall 0.18 + Coverage 0.59 → cân bằng, tốt cho production stable.
    \item \textbf{BPR}: Coverage 0.69 (cao nhất), nhưng Recall thấp.
    \item Lựa chọn mô hình phụ thuộc: \textbf{Accuracy vs Diversity}.
\end{itemize}
\end{frame}
%======================


%--------------------------------------------------
\begin{frame}{Hiệu quả của BERT Initialization}
\begin{itemize}
    \item Recall@10 +2.5\% so với ALS.
    \item Giảm cold-start cho sản phẩm mới.
    \item Embeddings có ngữ nghĩa từ đầu → tốt cho domain mỹ phẩm Việt Nam.
    \item SVD projection (1024 → 64 dims) giữ lại 64.9\% variance.
\end{itemize}
\end{frame}
%--------------------------------------------------



%--------------------------------------------------
\begin{frame}{Phân tích Cold-Augmented Models}
\begin{itemize}
    \item Cold-Aug models: Recall tốt nhưng NDCG thấp (~0.085).
    \item Nguyên nhân: Augmentation tạo noise → các item đúng nhưng thứ tự không tối ưu.
    \item Hướng cải tiến: weighted loss, augmentation chọn lọc.
\end{itemize}
\end{frame}
%--------------------------------------------------



%--------------------------------------------------
\begin{frame}{Phân khúc người dùng và hiệu suất hệ thống}
\textbf{User Segmentation:}
\begin{itemize}
    \item Trainable users (≥ 2 interactions): 8.7\%
    \item Cold-start users (< 2 interactions): 91.3\%
\end{itemize}

\textbf{System Performance:}
\begin{itemize}
    \item Data processing: < 1 phút
    \item ALS training (15 iters): 1–2 phút
    \item CF serving latency: 6.4 ms
    \item Hybrid serving latency: 43.3 ms
\end{itemize}
\end{frame}
%--------------------------------------------------



%======================
\begin{frame}{5.3 Thảo luận – Điểm mạnh}
\begin{itemize}
    \item Cải thiện lớn: BERT-ALS tăng 243.6\% so với baseline.
    \item Tất cả mô hình có ý nghĩa thống kê (p < 0.05).
    \item Nhiều lựa chọn model cho accuracy/diversity.
    \item Latency < 50ms – đáp ứng real-time.
    \item Hệ thống mở rộng tốt (modular, scalable).
\end{itemize}
\end{frame}
%======================


%--------------------------------------------------
\begin{frame}{5.3 Thảo luận – Hạn chế}
\begin{itemize}
    \item Sparsity cao: 91\% user cold-start.
    \item Rating skew: 95\% điểm là 5 sao.
    \item Coverage thấp cho BERT-ALS (21.2\%).
    \item Cold-Aug: NDCG thấp, cần tối ưu.
    \item Hybrid không hiệu quả cho trainable users.
\end{itemize}
\end{frame}
%--------------------------------------------------



%--------------------------------------------------
\begin{frame}{5.3 Thảo luận – Hướng cải tiến}
\begin{itemize}
    \item Contrastive learning cho embeddings.
    \item Knowledge graph cho cold-start.
    \item Cache similarity để giảm latency.
    \item Tối ưu chiến lược Cold-Augmentation.
    \item A/B testing trên real users.
\end{itemize}
\end{frame}
%--------------------------------------------------

%--------------------------------------------------
\begin{frame}[plain]
    \centering
    {\Huge \textbf{6. KẾT LUẬN VÀ HƯỚNG PHÁT TRIỂN}}
\end{frame}
%--------------------------------------------------



%=======================================================
%======================
\section{Kết luận và Hướng phát triển}
%======================

%--------------------------------------------------
\begin{frame}{6.1 Kết luận}
\textbf{Hệ thống đã giải quyết thành công các thách thức chính:}
\begin{enumerate}
    \item \textbf{Dữ liệu thưa (Sparsity):} Dual-path routing: CF cho trainable users, content-based fallback (PhoBERT + popularity) cho cold-start users. \textbf{100\% requests được phục vụ.}
    \item \textbf{Rating skew:} Sentiment-enhanced confidence score giúp phân biệt các tương tác thực sự tích cực.
    \item \textbf{Ngữ nghĩa tiếng Việt:} BERT Initialization (PhoBERT embeddings) cải thiện cold-start và chất lượng embeddings latent.
    \item \textbf{Kiến trúc production-ready:} Latency P95 < 100ms, availability ≥ 99.9\%, pipeline tự động đảm bảo data freshness và model updates.
\end{enumerate}
\end{frame}
%--------------------------------------------------


%--------------------------------------------------
\begin{frame}{6.2 Hướng phát triển}
\textbf{Ngắn hạn:}
\begin{itemize}
    \item A/B Testing Framework: đánh giá hiệu quả reranking (CTR, conversion rate).
    \item Real-time Features: tích hợp session-based signals (recently viewed, cart items) vào scoring.
\end{itemize}

\textbf{Trung hạn:}
\begin{itemize}
    \item Graph Neural Networks: mô hình hóa quan hệ user-item-attribute (heterogeneous graph).
    \item Multi-task Learning: đồng thời tối ưu click prediction và purchase prediction.
\end{itemize}

\textbf{Dài hạn:}
\begin{itemize}
    \item Reinforcement Learning: contextual bandits hoặc Q-learning để học policy gợi ý tối ưu theo thời gian thực.
    \item Explainable Recommendations: sinh giải thích tự nhiên cho từng gợi ý dựa trên LLM.
\end{itemize}
\end{frame}
%--------------------------------------------------
%--------------------------------------------------
\begin{frame}[plain]
    \centering
    {\Huge \textbf{Cảm ơn thầy và các bạn đã lắng nghe!}}
\end{frame}
%--------------------------------------------------


\end{document}

